% Options for packages loaded elsewhere
\PassOptionsToPackage{unicode}{hyperref}
\PassOptionsToPackage{hyphens}{url}
%
\documentclass[
  a4paper,
]{article}
\usepackage{amsmath,amssymb}
\usepackage{iftex}
\ifPDFTeX
  \usepackage[T1]{fontenc}
  \usepackage[utf8]{inputenc}
  \usepackage{textcomp} % provide euro and other symbols
\else % if luatex or xetex
  \usepackage{unicode-math} % this also loads fontspec
  \defaultfontfeatures{Scale=MatchLowercase}
  \defaultfontfeatures[\rmfamily]{Ligatures=TeX,Scale=1}
\fi
\usepackage{xcoffins}
\usepackage{lmodern}

\ifPDFTeX\else
% xetex/luatex font selection
    \setmainfont[]{David}
\fi
% Use upquote if available, for straight quotes in verbatim environments
\IfFileExists{upquote.sty}{\usepackage{upquote}}{}
\IfFileExists{microtype.sty}{% use microtype if available
  \usepackage[]{microtype}
  \UseMicrotypeSet[protrusion]{basicmath} % disable protrusion for tt fonts
}{}
\makeatletter
\@ifundefined{KOMAClassName}{% if non-KOMA class
  \IfFileExists{parskip.sty}{%
    \usepackage{parskip}
  }{% else
    \setlength{\parindent}{0pt}
    \setlength{\parskip}{6pt plus 2pt minus 1pt}}
}{% if KOMA class
  \KOMAoptions{parskip=half}}
\makeatother
\usepackage{xcolor}
\usepackage{color}
\usepackage{fancyvrb}
\newcommand{\VerbBar}{|}
\newcommand{\VERB}{\Verb[commandchars=\\\{\}]}
\DefineVerbatimEnvironment{Highlighting}{Verbatim}{commandchars=\\\{\}}
% Add ',fontsize=\small' for more characters per line
\newenvironment{Shaded}{}{}
\newcommand{\AlertTok}[1]{\textcolor[rgb]{1.00,0.00,0.00}{\textbf{#1}}}
\newcommand{\AnnotationTok}[1]{\textcolor[rgb]{0.38,0.63,0.69}{\textbf{\textit{#1}}}}
\newcommand{\AttributeTok}[1]{\textcolor[rgb]{0.49,0.56,0.16}{#1}}
\newcommand{\BaseNTok}[1]{\textcolor[rgb]{0.25,0.63,0.44}{#1}}
\newcommand{\BuiltInTok}[1]{\textcolor[rgb]{0.00,0.50,0.00}{#1}}
\newcommand{\CharTok}[1]{\textcolor[rgb]{0.25,0.44,0.63}{#1}}
\newcommand{\CommentTok}[1]{\textcolor[rgb]{0.38,0.63,0.69}{\textit{#1}}}
\newcommand{\CommentVarTok}[1]{\textcolor[rgb]{0.38,0.63,0.69}{\textbf{\textit{#1}}}}
\newcommand{\ConstantTok}[1]{\textcolor[rgb]{0.53,0.00,0.00}{#1}}
\newcommand{\ControlFlowTok}[1]{\textcolor[rgb]{0.00,0.44,0.13}{\textbf{#1}}}
\newcommand{\DataTypeTok}[1]{\textcolor[rgb]{0.56,0.13,0.00}{#1}}
\newcommand{\DecValTok}[1]{\textcolor[rgb]{0.25,0.63,0.44}{#1}}
\newcommand{\DocumentationTok}[1]{\textcolor[rgb]{0.73,0.13,0.13}{\textit{#1}}}
\newcommand{\ErrorTok}[1]{\textcolor[rgb]{1.00,0.00,0.00}{\textbf{#1}}}
\newcommand{\ExtensionTok}[1]{#1}
\newcommand{\FloatTok}[1]{\textcolor[rgb]{0.25,0.63,0.44}{#1}}
\newcommand{\FunctionTok}[1]{\textcolor[rgb]{0.02,0.16,0.49}{#1}}
\newcommand{\ImportTok}[1]{\textcolor[rgb]{0.00,0.50,0.00}{\textbf{#1}}}
\newcommand{\InformationTok}[1]{\textcolor[rgb]{0.38,0.63,0.69}{\textbf{\textit{#1}}}}
\newcommand{\KeywordTok}[1]{\textcolor[rgb]{0.00,0.44,0.13}{\textbf{#1}}}
\newcommand{\NormalTok}[1]{#1}
\newcommand{\OperatorTok}[1]{\textcolor[rgb]{0.40,0.40,0.40}{#1}}
\newcommand{\OtherTok}[1]{\textcolor[rgb]{0.00,0.44,0.13}{#1}}
\newcommand{\PreprocessorTok}[1]{\textcolor[rgb]{0.74,0.48,0.00}{#1}}
\newcommand{\RegionMarkerTok}[1]{#1}
\newcommand{\SpecialCharTok}[1]{\textcolor[rgb]{0.25,0.44,0.63}{#1}}
\newcommand{\SpecialStringTok}[1]{\textcolor[rgb]{0.73,0.40,0.53}{#1}}
\newcommand{\StringTok}[1]{\textcolor[rgb]{0.25,0.44,0.63}{#1}}
\newcommand{\VariableTok}[1]{\textcolor[rgb]{0.10,0.09,0.49}{#1}}
\newcommand{\VerbatimStringTok}[1]{\textcolor[rgb]{0.25,0.44,0.63}{#1}}
\newcommand{\WarningTok}[1]{\textcolor[rgb]{0.38,0.63,0.69}{\textbf{\textit{#1}}}}
\usepackage{longtable,booktabs,array}
\usepackage{calc} % for calculating minipage widths
% Correct order of tables after \paragraph or \subparagraph
\usepackage{etoolbox}
\makeatletter
\patchcmd\longtable{\par}{\if@noskipsec\mbox{}\fi\par}{}{}
\makeatother
% Allow footnotes in longtable head/foot
\IfFileExists{footnotehyper.sty}{\usepackage{footnotehyper}}{\usepackage{footnote}}
\makesavenoteenv{longtable}
\setlength{\emergencystretch}{3em} % prevent overfull lines
\providecommand{\tightlist}{%
  \setlength{\itemsep}{0pt}\setlength{\parskip}{0pt}}
\setcounter{secnumdepth}{-\maxdimen} % remove section numbering
\ifLuaTeX
\usepackage[bidi=basic]{babel}
\else
\usepackage[bidi=default]{babel}
\fi
\babelprovide[main,import]{hebrew}
\ifPDFTeX
\else
\babelfont{rm}[]{David}
\fi
% get rid of language-specific shorthands (see #6817):
\let\LanguageShortHands\languageshorthands
\def\languageshorthands#1{}
\ifLuaTeX
  \usepackage{selnolig}  % disable illegal ligatures
\fi
\ifPDFTeX
  \TeXXeTstate=1
  \newcommand{\RL}[1]{\beginR #1\endR}
  \newcommand{\LR}[1]{\beginL #1\endL}
  \newenvironment{RTL}{\beginR}{\endR}
  \newenvironment{LTR}{\beginL}{\endL}
\fi
\usepackage{bookmark}
\IfFileExists{xurl.sty}{\usepackage{xurl}}{} % add URL line breaks if available
\urlstyle{same}
\hypersetup{
  pdflang={he},
  hidelinks,
  pdfcreator={LaTeX via pandoc}}

\usepackage{fvextra}
\usepackage{bidi}


\author{}
\date{}

\begin{document}
\setRTL

\NewCoffin\Output   %Coffin to hold the others 
\NewCoffin\Callout % Callout definition ...
\NewCoffin\BackFrame % Background: green rectangle
\NewCoffin\SideRule  %lateral left border


\newcommand{\SetCallout}[2]{%
    \SetHorizontalCoffin\Output{} % It will be the reference point join the others  
    \SetVerticalCoffin\Callout{\linewidth-20pt}{\textbf{#1} #2}

    %% Make both \BackFrame & SideRule heights = height of Callout + 1*baselineskip
    \SetHorizontalCoffin\BackFrame{
      \begingroup
      \color{green!30!gray!15}\rule{\linewidth}{\CoffinTotalHeight\Callout + \baselineskip}
      \endgroup
    }    
    \SetHorizontalCoffin\SideRule{
      \begingroup  
      \color{green!50!black}\rule{3pt}{\CoffinTotalHeight\Callout +\baselineskip}
      \endgroup
    } %vertical side rule 

    %% Assembly Coffins
    \JoinCoffins*\Output[l,t]\Callout[l,t](10pt,-\baselineskip) %attach left-top corner of Callout to idem of Output
    \JoinCoffins*\Output[l,t]\SideRule[l,t] %attach left-top corner of  SideRule to idem of Output
    \JoinCoffins*\Output[l,t]\BackFrame[l,t] %attach left-top corner of BackFrame  to idem of Output
    %% Typeset ooutput
    \noindent\TypesetCoffin\Output % at the text insertion point. It is not a float.
    \vspace*{\CoffinTotalHeight\Callout}\bigskip %make some room for Output
}

\section{תרגיל בית 6}\label{ux5eaux5e8ux5d2ux5d9ux5dc-ux5d1ux5d9ux5ea-6}

\begin{longtable}[]{@{}rr@{}}
\toprule\noalign{}
& סטודנט א' \\
\midrule\noalign{}
\endhead
\bottomrule\noalign{}
\endlastfoot
\textbf{שם} & עידו פנג בנטוב \\
\textbf{ת''ז} & 322869140 \\
\textbf{דואר אלקטרוני} & ido.fang@campus.technion.ac.il \\
\end{longtable}

\begin{longtable}[]{@{}rr@{}}
\toprule\noalign{}
& סטודנט ב' \\
\midrule\noalign{}
\endhead
\bottomrule\noalign{}
\endlastfoot
\textbf{שם} & ניר קרל \\
\textbf{ת''ז} & 322437203 \\
\textbf{דואר אלקטרוני} & nir.karl@campus.technion.ac.il \\
\end{longtable}

\subsection{שאלה 1}\label{ux5e9ux5d0ux5dcux5d4-1}

\[I=\int_{0}^{6} \dfrac{x+1}{4x+3} \, \mathrm{d}x \]

\subsubsection{שאלה א'}\label{ux5e9ux5d0ux5dcux5d4-ux5d0}

\[\begin{aligned}
I&=\int_{0}^{6} \dfrac{x+1}{4x+3} \, \mathrm{d}x  \\[2ex]
&=\dfrac{1}{4}\int_{0}^{6} \dfrac{4x+4}{4x+3} \, \mathrm{d}x  \\[2ex]
&=\dfrac{1}{4}\int_{0}^{6} \dfrac{4x+3}{4x+3}+\dfrac{1}{4x+3} \, \mathrm{d}x \\[2ex]
  & =\dfrac{1}{4}\int_{0}^{6} 1 \, \mathrm{d}x +\dfrac{1}{4}\int_{0}^{6} \dfrac{1}{4x+3} \, \mathrm{d}x  \\[2ex]
&=\dfrac{1}{4}x\bigg|_{0}^{6} +\dfrac{1}{16}\ln(4x+3)\bigg|_{0}^{6} \\[2ex]
&=\dfrac{1}{4}\cdot 6+\dfrac{1}{16}\ln(27)-\dfrac{1}{16}\ln(3) \\[2ex]
&=\boxed {
1.63732653608
 }
\end{aligned}\]

\subsubsection{סעיף ב'}\label{ux5e1ux5e2ux5d9ux5e3-ux5d1}

האינטגרנד שלנו:

\[f(x)=\dfrac{x+1}{4x+3}\] נשים לב כי: \[a=0,\, \quad b=6\] וערכי
הפונקציה בנקודות אלו:
\[f(0)=\dfrac{1}{3},\, \quad f(3)=\dfrac{4}{15},\, \quad f(6)=\dfrac{7}{27}\]
ניוטון קוטס מסדר ראשון: \[I_{\text{trap}}=\dfrac{b-a}{2}[f(a)+f(b)]\]
נציב:
\[I_{\text{trap}}=3\left( \dfrac{1}{3}+\dfrac{7}{27} \right)=\boxed {
\dfrac{16}{9}
 }\] ניוטון קוטס מסדר שני:
\[I_{\text{Simp}}=\dfrac{b-a}{6}\left[ f(a)+4f\left( \dfrac{b+a}{2} \right)+f(b) \right]\]
נציב: \[\begin{aligned}
I_{\text{Simp}} & =1\cdot\left[ \dfrac{1}{3}+4\cdot \dfrac{4}{15}+\dfrac{7}{27} \right]=\boxed {
\dfrac{224}{135}
 }
\end{aligned}\]

השגיאה בפועל: \[\begin{aligned}
 & E_{\text{trap}}=I-I_{\text{trap}}=\boxed {
-0.14045124169
 } \\[2ex]
 & E_{\text{Simp}}=I-I_{\text{Simp}}=\boxed{ -0.02193272318}
\end{aligned}\]

השגיאה התאורטית: \[\begin{aligned}
 & E_{\text{trap}}=-\dfrac{f''(\xi)}{12}(b-a)^{3} \\[2ex]
 & E_{\text{Simp}}=-\dfrac{f^{(4)}(\xi)}{90}\left( \dfrac{b-a}{2} \right)^{5}
\end{aligned} \] כאשר \(0<\xi<6\).

נמצא את הנגזרות של \(f\): \[\begin{aligned}
 & f'(x)=\dfrac{4x+3-4(x+1)}{(4x+3)^{2}}=-\dfrac{1}{(4x+3)^{2}} =-(4x+3)^{-2}\\[1ex]
 & f''(x)=2(4x+3)^{-3} \\[1ex]
 & f'''(x)=-24(4x+3)^{-4} \\[1ex]
 & f^{(4)}(x)=384(4x+3)^{-5}
\end{aligned}\] הפונקציות \(f''\) ו-\(f^{(4)}\) הן פונקציות יורדות בקטע
הנתון. לכן נקבל ערך מקסימלי בקצוות. \[\begin{aligned}
 & f''(0)=\dfrac{2}{27} &  & f''(6)=1.0162\cdot 10^{-4}\\
 & f^{(4)}(0)=\dfrac{128}{81} &  & f^{(4)}(6)=2.676\cdot 10^{-5}
\end{aligned}\] נסיק כי החסמים העליונים והתחתונים עבור שתי השיטות הן:
\[\begin{gathered}
-\dfrac{4}{3}=-\dfrac{f''(0)}{12}\cdot 6^{3}<E_{\text{trap}}<-\dfrac{f''(6)}{12}\cdot 6^{3}=-0.00183 \\[2ex]
-4.2666<-\dfrac{f^{(4)}(0)}{90}\cdot 3^{5}<E_{\text{Simp}}<-\dfrac{f^{(4)}(6)}{90}\cdot 3^{5}=-7.2252\cdot 10^{-5}
\end{gathered}\] אכן השגיאות שלנו נמצאים בטווח של השגיאה התאורטית.

\subsubsection{סעיפים ג', ד
ו-ה'}\label{ux5e1ux5e2ux5d9ux5e4ux5d9ux5dd-ux5d2-ux5d3-ux5d5-ux5d4}

\begin{Shaded}
\begin{Highlighting}[]
\ImportTok{import}\NormalTok{ numpy }\ImportTok{as}\NormalTok{ np}
\ImportTok{import}\NormalTok{ sympy }\ImportTok{as}\NormalTok{ sp}
\ImportTok{import}\NormalTok{ matplotlib.pyplot }\ImportTok{as}\NormalTok{ plt}

\KeywordTok{def}\NormalTok{ composite\_trapezoidal(f, a, b, n):}
\NormalTok{    h }\OperatorTok{=}\NormalTok{ (b }\OperatorTok{{-}}\NormalTok{ a) }\OperatorTok{/}\NormalTok{ n}
\NormalTok{    s }\OperatorTok{=}\NormalTok{ f(a) }\OperatorTok{+}\NormalTok{ f(b)}
    \ControlFlowTok{for}\NormalTok{ i }\KeywordTok{in} \BuiltInTok{range}\NormalTok{(}\DecValTok{1}\NormalTok{, n):}
\NormalTok{        s }\OperatorTok{+=} \DecValTok{2} \OperatorTok{*}\NormalTok{ f(a }\OperatorTok{+}\NormalTok{ i }\OperatorTok{*}\NormalTok{ h)}
    \ControlFlowTok{return}\NormalTok{ s }\OperatorTok{*}\NormalTok{ h }\OperatorTok{/} \DecValTok{2}

\KeywordTok{def}\NormalTok{ composite\_simpsons(f, a, b, n):}
\NormalTok{    h }\OperatorTok{=}\NormalTok{ (b }\OperatorTok{{-}}\NormalTok{ a) }\OperatorTok{/}\NormalTok{ n}
\NormalTok{    s }\OperatorTok{=}\NormalTok{ f(a) }\OperatorTok{+}\NormalTok{ f(b)}
    \ControlFlowTok{for}\NormalTok{ i }\KeywordTok{in} \BuiltInTok{range}\NormalTok{(}\DecValTok{1}\NormalTok{, n):}
        \ControlFlowTok{if}\NormalTok{ i }\OperatorTok{\%} \DecValTok{2} \OperatorTok{==} \DecValTok{0}\NormalTok{:}
\NormalTok{            s }\OperatorTok{+=} \DecValTok{2} \OperatorTok{*}\NormalTok{ f(a }\OperatorTok{+}\NormalTok{ i }\OperatorTok{*}\NormalTok{ h)}
        \ControlFlowTok{else}\NormalTok{:}
\NormalTok{            s }\OperatorTok{+=} \DecValTok{4} \OperatorTok{*}\NormalTok{ f(a }\OperatorTok{+}\NormalTok{ i }\OperatorTok{*}\NormalTok{ h)}
    \ControlFlowTok{return}\NormalTok{ s }\OperatorTok{*}\NormalTok{ h }\OperatorTok{/} \DecValTok{3}

\KeywordTok{def}\NormalTok{ richardson\_extrapolation(f, a, b, n, method):}
\NormalTok{    I1 }\OperatorTok{=}\NormalTok{ method(f, a, b, n)}
\NormalTok{    I2 }\OperatorTok{=}\NormalTok{ method(f, a, b, }\DecValTok{2}\OperatorTok{*}\NormalTok{n)}
    \ControlFlowTok{return}\NormalTok{ (}\DecValTok{4}\OperatorTok{*}\NormalTok{I2 }\OperatorTok{{-}}\NormalTok{ I1) }\OperatorTok{/} \DecValTok{3}

\NormalTok{f }\OperatorTok{=} \KeywordTok{lambda}\NormalTok{ x: (x}\OperatorTok{+}\DecValTok{1}\NormalTok{)}\OperatorTok{/}\NormalTok{(}\DecValTok{4}\OperatorTok{*}\NormalTok{x}\OperatorTok{+}\DecValTok{3}\NormalTok{)}
\NormalTok{a }\OperatorTok{=} \DecValTok{0}
\NormalTok{b }\OperatorTok{=} \DecValTok{6}

\NormalTok{points }\OperatorTok{=}\NormalTok{ [}\DecValTok{21}\NormalTok{,}\DecValTok{41}\NormalTok{,}\DecValTok{81}\NormalTok{,}\DecValTok{161}\NormalTok{]}
\NormalTok{n }\OperatorTok{=}\NormalTok{ np.array(points) }\OperatorTok{{-}} \DecValTok{1}

\NormalTok{x }\OperatorTok{=}\NormalTok{ sp.symbols(}\StringTok{\textquotesingle{}x\textquotesingle{}}\NormalTok{)}
\NormalTok{fsym }\OperatorTok{=}\NormalTok{ (x}\OperatorTok{+}\DecValTok{1}\NormalTok{)}\OperatorTok{/}\NormalTok{(}\DecValTok{4}\OperatorTok{*}\NormalTok{x}\OperatorTok{+}\DecValTok{3}\NormalTok{)}
\NormalTok{I\_real }\OperatorTok{=}\NormalTok{ sp.integrate(fsym, (x, }\DecValTok{0}\NormalTok{, }\DecValTok{6}\NormalTok{))}
\NormalTok{I\_real }\OperatorTok{=} \BuiltInTok{float}\NormalTok{(I\_real)}

\NormalTok{E\_trap }\OperatorTok{=}\NormalTok{ [}\BuiltInTok{abs}\NormalTok{(I\_real }\OperatorTok{{-}}\NormalTok{ composite\_trapezoidal(f, a, b, i)) }\ControlFlowTok{for}\NormalTok{ i }\KeywordTok{in}\NormalTok{ n]}
\NormalTok{E\_simp }\OperatorTok{=}\NormalTok{ [}\BuiltInTok{abs}\NormalTok{(I\_real }\OperatorTok{{-}}\NormalTok{ composite\_simpsons(f, a, b, i)) }\ControlFlowTok{for}\NormalTok{ i }\KeywordTok{in}\NormalTok{ n]}
\NormalTok{E\_rich }\OperatorTok{=}\NormalTok{ [}\BuiltInTok{abs}\NormalTok{(I\_real }\OperatorTok{{-}}\NormalTok{ richardson\_extrapolation(f, a, b, i, composite\_simpsons)) }\ControlFlowTok{for}\NormalTok{ i }\KeywordTok{in}\NormalTok{ n]}

\NormalTok{plt.plot(n, E\_trap, label}\OperatorTok{=}\StringTok{"Trapezoidal rule"}\NormalTok{)}
\NormalTok{plt.plot(n, E\_simp, label}\OperatorTok{=}\StringTok{"Simpson\textquotesingle{}s rule"}\NormalTok{)}
\NormalTok{plt.plot(n, E\_rich, label}\OperatorTok{=}\StringTok{"Richardson extrapolation"}\NormalTok{)}
\NormalTok{plt.yscale(}\StringTok{\textquotesingle{}log\textquotesingle{}}\NormalTok{)}
\NormalTok{plt.xscale(}\StringTok{\textquotesingle{}log\textquotesingle{}}\NormalTok{)}
\NormalTok{plt.xlabel(}\StringTok{\textquotesingle{}n\textquotesingle{}}\NormalTok{)}
\NormalTok{plt.ylabel(}\StringTok{\textquotesingle{}Error\textquotesingle{}}\NormalTok{)}
\NormalTok{plt.legend()}
\NormalTok{plt.show()}
\end{Highlighting}
\end{Shaded}

\begin{quote}
{[}!TODO{]} להוסיף גרף
\end{quote}

\begin{quote}
{[}!TODO{]} להוסיף לגבי תאוריה
\end{quote}

\subsection{תרגיל 2}\label{ux5eaux5e8ux5d2ux5d9ux5dc-2}

\subsubsection{סעיף א'}\label{ux5e1ux5e2ux5d9ux5e3-ux5d0}

נשתמש בשיטת המקדמים החופשיים:
\[f(x)=1\longrightarrow \int_{0}^{1} \frac{1}{\sqrt{x} \cdot \sqrt{1-x}} \; dx=\pi=A_{1}+A_{2}\]
\[f(x)=x\longrightarrow \int_{0}^{1} \frac{x}{\sqrt{x} \cdot \sqrt{1-x}} \; dx=\frac{\pi}{2}=\frac{1}{3}A_{1}+\frac{2}{3}A_{2} \]
\[\begin{array}{rl} \\
\mathrm{I} :&\pi=A_{1}+A_{2} \\
\mathrm{II} :&\frac{\pi}{2}=\frac{1}{3}A_{1}+\frac{2}{3}A_{2} \\
3\cdot \mathrm{II} -\mathrm{I} :&\frac{\pi}{2}=A_{2} \\
A_{2}\to \mathrm{I} :& \pi=A_{1}+\frac{\pi}{2}\Longrightarrow A_{1}=\frac{\pi}{2}
\end{array}\] \[\boxed{A_{1}=A_{2}=\frac{\pi}{2}}\]

\subsubsection{סעיף ב'}\label{ux5e1ux5e2ux5d9ux5e3-ux5d1-1}

\[f(x)=\sqrt{x}\cdot \sqrt{(1+x)}\]
\[I=\int_{0}^{1} f(x) \; dx =\frac{\pi}{2}f\left( \frac{1}{3} \right)+\frac{\pi}{2}f\left( \frac{2}{3} \right)+E\]
\[\frac{\pi}{2}\sqrt{\frac{1}{3}}\cdot \sqrt{\frac{4}{3}}+\frac{\pi}{2}\sqrt{\frac{2}{3}}\cdot \sqrt{\frac{5}{3}}+E=\frac{\pi}{3}+\frac{\pi}{2}\cdot \frac{\sqrt{10}}{3}+E\]
\[\boxed{I=2.70296+E}\]

\subsubsection{סעיף ג'}\label{ux5e1ux5e2ux5d9ux5e3-ux5d2}

בקטע \([0,\;1]\) הפונקציה \(f(x)=\frac{1}{\sqrt{x(1-x)}}\) לא רציפה בכל
נק', ולכן גם לא גזירה בכל נק'. לכן אי אפשר לעשות לה פיתוח מקלורן - ומכאן
שאי אפשר לחסום את השגיאה.

\subsection{תרגיל 3}\label{ux5eaux5e8ux5d2ux5d9ux5dc-3}

\subsubsection{סעיף א'}\label{ux5e1ux5e2ux5d9ux5e3-ux5d0-1}

\[\begin{aligned}
I & =\int_{0}^{1.2} e^{-x} \, \mathrm{d}x \\[1ex]
 &  =-e^{-x}\bigg|_{0}^{1.2} \\[1ex]
& =e^{0}-e^{-1.2} \\[1ex]
 & =\boxed {
0.69880578809
 }
\end{aligned}\]

\subsubsection{סעיף ב'}\label{ux5e1ux5e2ux5d9ux5e3-ux5d1-2}

עבור 2 צמתי אינטגרציה: \[x_{1}=0,\; x_{2}=1.2\] \[\begin{aligned}
\tilde{I}  & =\;\frac{h}{2}\Big[f(x_{2})+f(x_{1})\Big]\\[1ex]
&=\underbrace{0.6}_{1.2/2}\cdot (1+0.3012)\\[1ex]
 & =0.78071652715
\end{aligned}\] ולכן השגיאה שלנו: \[\begin{aligned}
E_{\text{trap}} & =0.69880578809-0.78071652715 \\[1ex]
 & =\boxed{-0.08191073906}
\end{aligned}\] עבור 3 צמתי אינטגרציה:

\[x_{1}=0,\; x_{2}=0.6,\; x_{3}=1.2\] \[\begin{aligned}
\tilde{I}  & =\;\frac{h}{3}\Big[f(x_{1})+4f(x_{2})+f(x_{3})\Big] \\[1ex]
&=\underbrace{0.2}_{0.6/3}\Big[1+4\cdot 0.5488+0.3012\Big] \\[1ex]
 & =0.69928815126
\end{aligned}\] ולכן השגיאה שלנו: \[\begin{aligned}
E_{\text{Simp}} & =0.69880578809-0.69928815126 \\[1ex]
 & =\boxed{-0.00048236317}
\end{aligned}\]

\subsubsection{סעיף ג'}\label{ux5e1ux5e2ux5d9ux5e3-ux5d2-1}

נעביר לתחום אינטגרציה \([-1,1]\): \[\begin{aligned}
 & x=T(t)=\frac{(1.2-0)t+(1.2+0)}{2}=0.6t+0.6 \\[2ex]
 & \mathrm{d}x=0.6\mathrm{d}t
\end{aligned}\] נציב בנוסחה של אינטגרציית גאוס:
\[\int_{0}^{1.2} f(x) \, \mathrm{d}x =\int_{-1}^{1} 0.6\cdot f(0.6t+0.6) \, \mathrm{d}t =\int_{-1}^{1}0.6 e^{-(0.6t+0.6)} \, \mathrm{d}t\]
נסמן: \[g(t)=0.6e^{-(0.6t+0.6)}\] ולכן:
\[\int_{0}^{1.2} f(x) \, \mathrm{d}x =\int_{-1}^{1} g(t) \, \mathrm{d}t\approx {A}_{1}g({t}_{1})+{A}_{2}g({t}_{2}) \]
מטבלת שורשי לג'נדר: \[\begin{aligned}
 & t_{1}=-0.57735 &  &  A_{1}=1 \\[1ex]
 & t_{2}=0.57735 &  &  A_{2}=1
\end{aligned}\] נוכל כעת לשער את ערך האינטגרל: \[\begin{aligned}
\tilde{I}  & = {A_{1}}g(t_{1})+{A_{2}}_{}g(t_{2})\\[2ex]
&=0.6\cdot e^{-0.6}\cdot \left( e^{0.6\cdot 0.57735}+e^{-0.6\cdot 0.57735}\right) \\[2ex]
 & =0.69848509187
\end{aligned}\] נסיק כי השגיאה: \[\begin{aligned}
  E & =0.69880578809-0.69848509187 \\
 & =\boxed{0.00032069622}
\end{aligned}\]

\[\left| E_{\text{gauss}}^{(2)} \right|<\left| E_{\text{trap}} \right|\]
נשים לב שבסעיף ב' קיבלנו חסם שלילי, כלומר בשיטה מסעיף ב' קיבלנו שטח קטן
מהשטח האמיתי. לעומת זאת בשיטה בסעיף זה החסם הוא חיובי, כלומר תמיד נקבל
שטח גדול מהשטח האמיתי.

נבדוק את השגיאה התיאורטית לפי שיטת גאוס-לז'נדר. עבור \(0<\xi<1.2\):
\[E=\frac{2^{(2n+1)}(n!)^{4}}{(2n+1)[(2n)!]^{3}}g^{(2n)}(\xi)=\frac{2^{5}\cdot 2^{4}}{5\cdot 24^{3}}\cdot g^{(4)}(\xi)=0.0074\cdot g^{(4)}(\xi)\]
נשים לב שהנגזרת הרביעית היא: \[g^{(4)}(x)=0.0426756 \cdot e^{-0.6 t}\]
ולכן החסם העליון והתחתון הם: \[\begin{aligned}
&0.0074\cdot g^{(4)}(1)=0.00031579944 \cdot e^{-0.6 \cdot 1}=0.000288857 \\
&0.0074\cdot g^{(4)}(-1)=0.00031579944 \cdot e^{-0.6 \cdot (-1)}=0.000576
\end{aligned}\] \[\boxed{0.000173\leq E=0.00032069622\leq 0.000576}\]

קיבלנו ש- \(E\) בפועל אכן בטווח ששיערנו שהוא יהיה.

\subsubsection{סעיף ד'}\label{ux5e1ux5e2ux5d9ux5e3-ux5d3}

מסעיף קודם: \[g(t)=0.6e^{-(0.6t+0.6)}\] שורשי לג'נדר: \[\begin{aligned}
 & t_{1,2}=\pm 0.774597 &  & A_{1,2}=\frac{5}{9} \\
 & t_{3}=0.774597 &  &  A_{3}=\frac{8}{9}
\end{aligned}\] נוכל כעת לשער את ערך האינטגרל: \[\begin{aligned}
\tilde{I}  & =A_{1}g(x_{1})+A_{2}g(x_{2})+A_{3}g(x_{3}) \\[2ex]
&= 0.6\cdot e^{-0.6}\cdot \Big(\frac{5}{9}\cdot e^{0.6\cdot 0 }+\frac{8}{9}\cdot  e^{0.6\cdot 0.774597 }+\frac{8}{9}\cdot  e^{-0.6\cdot 0.774597 } \Big) \\[2ex]
&=0.9880483684
\end{aligned}\] נסיק כי השגיאה: \[\begin{aligned}
E & =0.69880578809-0.69880483684 \\
 & =\boxed{0.9512\cdot 10^{-6}}
\end{aligned}\]
\[\left| E_{\text{gauss}}^{(3)} \right|<\left| E_{\text{trap}} \right|\]
נשים לב שבסעיף ב' קיבלנו חסם שלילי, כלומר בשיטה מסעיף ב' קיבלנו שטח קטן
מהשטח האמיתי. לעומת זאת בשיטה בסעיף זה החסם הוא חיובי, כלומר תמיד נקבל
שטח גדול מהשטח האמיתי.

נבדוק את השגיאה התיאורטית לפי שיטת גאוס-לז'נדר. עבור \(0<\xi<1.2\):
\[E=\frac{2^{(2n+1)}(n!)^{4}}{(2n+1)[(2n)!]^{3}}g^{(2n)}(\xi)=\frac{2^{7}\cdot 6^{4}}{7\cdot 720^{3}}\cdot g^{(6)}(\xi)=6.3492\cdot 10^{-5}\cdot g^{(4)}(\xi)\]
נשים לב שהנגזרת הרביעית היא: \[g^{(6)}(t)=0.0153632\cdot e^{-0.6t}\]
ולכן החסם העליון והתחתון הם:

\[\begin{aligned}
&6.3492\cdot 10^{-5}\cdot g^{(6)}(1)=0.97544\cdot 10^{-6}\cdot e^{-0.6\cdot 1}= 0.5354\cdot 10^{-6}\\
&6.3492\cdot 10^{-5}\cdot g^{(6)}(-1)=0.97544\cdot 10^{-6}\cdot e^{-0.6\cdot (-1)}=1.777\cdot 10^{-6}
\end{aligned}\]
\[\boxed{0.5354\cdot 10^{-6}\leq E=0.9512\cdot 10^{-6}\leq 1.777\cdot 10^{-6}}\]
קיבלנו ש- \(E\) בפועל אכן בטווח ששיערנו שהוא יהיה.

\end{document}
